\chapter{Oracle-XML}
\section{Créer  des  tables  pour  enregistrer  vos  documents  XML  en  utilisant  le  différentes  options  de  stockage.}
\subsection{CLOB}
\begin{minted}[linenos, breaklines]{sql}
CREATE TABLE tweet_CLOB (
       text_content varchar(20),
       xml_content XMLTYPE)
       XMLTYPE xml_content STORE AS CLOB;
\end{minted}
       
\subsection{BINARY}
\begin{minted}[linenos, breaklines]{sql}
CREATE TABLE tweet_BINARY (
       text_content varchar(20),
       xml_content XMLTYPE)
       XMLTYPE xml_content STORE AS BINARY XML;
\end{minted}

\section{Enregistrez votre document XML pour les Tweets dans la base avec l’instruction suivante (attention auxcaractères " ‘ " et " ’ " dans le copier-coller.)}
\subsection{CLOB}
\begin{minted}[linenos, breaklines]{sql}
INSERT INTO tweet_CLOB(text_content,xml_content)VALUES('tweet.xml', sys.xmltype.createxml('
<tweeter>
  <users>
    <user id="U41">
      <user_name>
	basil_dalie
      </user_name>
      <first_name>
	Basil
      </first_name>
      <last_name>
	Dalie
      </last_name>
      <profile>
	https://twitter.com/basil_dalie
      </profile>
    </user>
    <user id="U02">
      <user_name>
	Cicero
      </user_name>
      <first_name>
	Ci
      </first_name>
      <last_name>
	Cero
      </last_name>
      <profile>
	https://twitter.com/Cicero
      </profile>      
    </user>
    <user id="U43">
      <user_name>
	alex_not
      </user_name>
      <first_name>
	Alex
      </first_name>
      <last_name>
	Not
      </last_name>
      <profile>
	https://twitter.com/alex_n
      </profile>
    </user>
  </users>
  <tweets>
    <tweet id="T42" author_ref="U41">
      <header>
	<date>
	  1758312000
	</date>
	<timezone>
	  <standard>
	    UTC
	  </standard>
	  <offset>
	    1
	  </offset>
	</timezone>
	<location>
	  <latitude>
	    3.876716
	  </latitude>
	  <longitude>
	    43.610769
	  </longitude>
	  <city>
	    Montpellier
	  </city>
	  <country>
	    France
	  </country>
	</location>
	<language>
	  Latin
	</language>
	<retweets>
	  <retweet ref="T04" />
	  <retweet ref="T05" />
	  <retweet ref="T03" />
	</retweets>
	<answers>
	  <answer ref="T02" />
	</answers>
	<operating_system>
	  Linux x86_64
	</operating_system>
	<images>
	  <image id="I01">
	    https://images2.minutemediacdn.com/image/upload/c_crop,h_1193,w_2121,x_0,y_64/f_auto,q_auto,w_1100/v1565279671/shape/mentalfloss/578211-gettyimages-542930526.jpg
	  </image>
	  <image id ="I02">
	    https://website.com/image.jpg
	  </image>
	</images>
	<videos>
	  <video id="V01">
	    https://website.com/video.avi
	  </video>
	</videos>
      </header>
      <body>
        <hashtag>
          #I3XML
        </hashtag>
	<text>
	  Lorem ipsum dolor sit amet, consectetur adipiscing elit, sed do eiusmod tempor incididunt ut labore et dolore magna aliqua. Ut enim ad minim veniam, quis nostrud exercitation ullamco laboris nisi ut aliquip ex ea commodo consequat. Duis aute irure dolor in reprehenderit in voluptate velit esse cillum dolore eu fugiat nulla pariatur. Excepteur sint occaecat cupidatat non proident, sunt in culpa qui officia deserunt
	</text>
	<text>
	  anim id est laborum
	</text>
	<image_ref ref="I01" />
	<user_ref ref="U02">
	  \@Cicero
	</user_ref>
	<text>
	  Sed ut perspiciatis unde omnis iste natus error sit voluptatem accusantium doloremque laudantium, totam rem aperiam, eaque ipsa quae ab illo inventore veritatis et quasi architecto beatae vitae dicta sunt explicabo. Nemo enim ipsam voluptatem quia voluptas sit aspernatur aut odit aut fugit, sed quia consequuntur magni dolores eos qui ratione voluptatem sequi nesciunt. Neque porro quisquam est, qui dolorem ipsum quia dolor sit amet, consectetur, adipisci velit, sed quia non numquam eius modi tempora incidunt ut labore et dolore magnam aliquam quaerat voluptatem. Ut enim ad minima veniam, quis nostrum exercitationem ullam corporis suscipit laboriosam, nisi ut aliquid ex ea commodi consequatur? Quis autem vel eum iure reprehenderit qui in ea voluptate velit esse quam nihil molestiae consequatur, vel illum qui dolorem eum fugiat quo voluptas nulla pariatur?
	</text>
	<video_ref ref="V01" />
      </body>
    </tweet>
  </tweets>
</tweeter>'));
\end{minted}

\subsection{BINARY}
\begin{minted}[linenos, breaklines]{sql}
INSERT INTO tweet_BINARY(text_content,xml_content)VALUES('tweet.xml', sys.xmltype.createxml('
<tweeter>
  <users>
    <user id="U41">
      <user_name>
	basil_dalie
      </user_name>
      <first_name>
	Basil
      </first_name>
      <last_name>
	Dalie
      </last_name>
      <profile>
	https://twitter.com/basil_dalie
      </profile>
    </user>
    <user id="U02">
      <user_name>
	Cicero
      </user_name>
      <first_name>
	Ci
      </first_name>
      <last_name>
	Cero
      </last_name>
      <profile>
	https://twitter.com/Cicero
      </profile>      
    </user>
    <user id="U43">
      <user_name>
	alex_not
      </user_name>
      <first_name>
	Alex
      </first_name>
      <last_name>
	Not
      </last_name>
      <profile>
	https://twitter.com/alex_n
      </profile>
    </user>
  </users>
  <tweets>
    <tweet id="T42" author_ref="U41">
      <header>
	<date>
	  1758312000
	</date>
	<timezone>
	  <standard>
	    UTC
	  </standard>
	  <offset>
	    1
	  </offset>
	</timezone>
	<location>
	  <latitude>
	    3.876716
	  </latitude>
	  <longitude>
	    43.610769
	  </longitude>
	  <city>
	    Montpellier
	  </city>
	  <country>
	    France
	  </country>
	</location>
	<language>
	  Latin
	</language>
	<retweets>
	  <retweet ref="T04" />
	  <retweet ref="T05" />
	  <retweet ref="T03" />
	</retweets>
	<answers>
	  <answer ref="T02" />
	</answers>
	<operating_system>
	  Linux x86_64
	</operating_system>
	<images>
	  <image id="I01">
	    https://images2.minutemediacdn.com/image/upload/c_crop,h_1193,w_2121,x_0,y_64/f_auto,q_auto,w_1100/v1565279671/shape/mentalfloss/578211-gettyimages-542930526.jpg
	  </image>
	  <image id ="I02">
	    https://website.com/image.jpg
	  </image>
	</images>
	<videos>
	  <video id="V01">
	    https://website.com/video.avi
	  </video>
	</videos>
      </header>
      <body>
	<text>
	  Lorem ipsum dolor sit amet, consectetur adipiscing elit, sed do eiusmod tempor incididunt ut labore et dolore magna aliqua. Ut enim ad minim veniam, quis nostrud exercitation ullamco laboris nisi ut aliquip ex ea commodo consequat. Duis aute irure dolor in reprehenderit in voluptate velit esse cillum dolore eu fugiat nulla pariatur. Excepteur sint occaecat cupidatat non proident, sunt in culpa qui officia deserunt
	</text>
	<text>
	  anim id est laborum
	</text>
	<image_ref ref="I01" />
	<user_ref ref="U02">
	  \@Cicero
	</user_ref>
	<text>
	  Sed ut perspiciatis unde omnis iste natus error sit voluptatem accusantium doloremque laudantium, totam rem aperiam, eaque ipsa quae ab illo inventore veritatis et quasi architecto beatae vitae dicta sunt explicabo. Nemo enim ipsam voluptatem quia voluptas sit aspernatur aut odit aut fugit, sed quia consequuntur magni dolores eos qui ratione voluptatem sequi nesciunt. Neque porro quisquam est, qui dolorem ipsum quia dolor sit amet, consectetur, adipisci velit, sed quia non numquam eius modi tempora incidunt ut labore et dolore magnam aliquam quaerat voluptatem. Ut enim ad minima veniam, quis nostrum exercitationem ullam corporis suscipit laboriosam, nisi ut aliquid ex ea commodi consequatur? Quis autem vel eum iure reprehenderit qui in ea voluptate velit esse quam nihil molestiae consequatur, vel illum qui dolorem eum fugiat quo voluptas nulla pariatur?
	</text>
	<video_ref ref="V01" />
      </body>
    </tweet>
  </tweets>
</tweeter>'));
\end{minted}

\section{À  l’aide  des  commandes  suivantes,  exécuter5requêtes  Xquery  et5requêtes  XQuery  parmi  celles  quevous avez proposées pour le TP précédent.}
\subsection{CLOB}
\subsubsection{Les noms des auteurs des tweets.}
\begin{minted}[linenos, breaklines]{xquery}
  /tweeter/users/user[@id = /tweeter/tweets/tweet/@author_ref]/concat(first_name, ' ', last_name)
\end{minted}
\begin{minted}[linenos, breaklines]{sql}
  SELECT EXTRACT(xml_content,'/tweeter/users/user[@id=/tweeter/tweets/tweet/@author_ref]/user_name') AS nom FROM tweet_CLOB;
\end{minted}

\subsubsection{Les tweets de l’utilisateur dont l’id est "u41".}
\begin{minted}[linenos, breaklines]{xquery}
  /tweeter/tweets/tweet[@author_ref='U41']
\end{minted}
\begin{minted}[linenos, breaklines]{sql}
  SELECT EXTRACT(xml_content,'/tweeter/tweets/tweet[@author_ref=''U41'']') AS tweet FROM tweet_CLOB;
\end{minted}

\subsubsection{Les tweets contenants l’hashtag “\#I3XML"}
\begin{minted}[linenos, breaklines]{xquery}
  /tweeter/tweets/tweet/body/hashtag[contains(text(), '#I3XML')]/ancestor::tweet	
\end{minted}
\begin{minted}[linenos, breaklines]{sql}
  SELECT EXTRACT(xml_content,'/tweeter/tweets/tweet/body/hashtag[contains(text(), ''#I3XML'')]') FROM tweet_CLOB;
\end{minted}

\subsubsection{Les retweets du tweet dont l’id est "t42".}
\begin{minted}[linenos, breaklines]{xquery}
  /tweeter/tweets/tweet[@id = /tweeter/tweets/tweet[@id = "T42"]/header/retweets/retweet/@ref]
\end{minted}
\begin{minted}[linenos, breaklines]{sql}
  SELECT EXTRACT(xml_content, '/tweeter/tweets/tweet[@id = /tweeter/tweets/tweet[@id = "T42"]/header/retweets/retweet/@ref]') AS tweet FROM tweet_CLOB;
\end{minted}

\subsubsection{Les tweet sans hashtags.}
\begin{minted}[linenos, breaklines]{xquery}
  /tweeter/tweets/tweet/body[count(hashtag) = 0]/parent::tweet
\end{minted}
\begin{minted}[linenos, breaklines]{sql}
  SELECT EXTRACT(xml_content,'/tweeter/tweets/tweet/body[count(hashtag) = 0]/parent::tweet') FROM tweet_CLOB;
\end{minted}

\subsection{BINARY}
\subsubsection{Les noms des auteurs des tweets.}
\begin{minted}[linenos, breaklines]{xquery}
  /tweeter/users/user[@id = /tweeter/tweets/tweet/@author_ref]/concat(first_name, ' ', last_name)
\end{minted}
\begin{minted}[linenos, breaklines]{sql}
  SELECT EXTRACT(xml_content,'/tweeter/users/user[@id=/tweeter/tweets/tweet/@author_ref]/user_name') AS nom FROM tweet_BINARY;
\end{minted}

\subsubsection{Les tweets de l’utilisateur dont l’id est "u41".}
\begin{minted}[linenos, breaklines]{xquery}
  /tweeter/tweets/tweet[@author_ref='U41']
\end{minted}
\begin{minted}[linenos, breaklines]{sql}
  SELECT EXTRACT(xml_content,'/tweeter/tweets/tweet[@author_ref=''U41'']') AS tweet FROM tweet_BINARY;
\end{minted}

\subsubsection{Les tweets contenants l’hashtag “\#I3XML"}
\begin{minted}[linenos, breaklines]{xquery}
  /tweeter/tweets/tweet/body/hashtag[contains(text(), '#I3XML')]/ancestor::tweet	
\end{minted}
\begin{minted}[linenos, breaklines]{sql}
  SELECT EXTRACT(xml_content,'/tweeter/tweets/tweet/body/hashtag[contains(text(), ''#I3XML'')]') FROM tweet_BINARY;
\end{minted}

\subsubsection{Les retweets du tweet dont l’id est "t42".}
\begin{minted}[linenos, breaklines]{xquery}
  /tweeter/tweets/tweet[@id = /tweeter/tweets/tweet[@id = "T42"]/header/retweets/retweet/@ref]
\end{minted}
\begin{minted}[linenos, breaklines]{sql}
  SELECT EXTRACT(xml_content, '/tweeter/tweets/tweet[@id = /tweeter/tweets/tweet[@id = "T42"]/header/retweets/retweet/@ref]') AS tweet FROM tweet_BINARY;
\end{minted}

\subsubsection{Les tweet sans hashtags.}
\begin{minted}[linenos, breaklines]{xquery}
  /tweeter/tweets/tweet/body[count(hashtag) = 0]/parent::tweet
\end{minted}
\begin{minted}[linenos, breaklines]{sql}
  SELECT EXTRACT(xml_content,'/tweeter/tweets/tweet/body[count(hashtag) = 0]/parent::tweet') FROM tweet_BINARY;
\end{minted}

\subsection{CLOB}
\subsubsection{Créez une liste de paires tweet-auteur, avec chaque paire contenue dans un element result.}
\begin{minted}[linenos, breaklines]{sql}
  SELECT XMLQUERY('for $auteur in /tweeter/users/user 
  for  $tweet in /tweeter/tweets/tweet
  where $auteur/@id = $tweet/@author_ref  
  return
  <result>
  { $tweet }
  { $auteur }
  </result>' PASSING xml_content RETURNING CONTENT)FROM tweet_CLOB;
\end{minted}

\subsubsection{Listez les utilisateurs de la plateforme en ordre alphabétique.}
\begin{minted}[linenos, breaklines]{sql}
  SELECT XMLQUERY('for $user in tweeter/users/user
  order by upper-case($user/user_name/text()) ascending
  return  $user' PASSING xml_content RETURNING CONTENT)FROM tweet_CLOB;
\end{minted}


\subsubsection{Pour chaque utilisateur, listez le nom de l’utilisateur et la date de tous ses tweets, le tout regroupé dansun élément result.}
\begin{minted}[linenos, breaklines]{sql}
  SELECT XMLQUERY('for $auteur in /tweeter/users/user
  return
  <result>
    <nom>
      { $auteur/user_name/text()}
    </nom> 
    {
      for $tweet in /tweeter/tweets/tweet
      where $auteur/@id = $tweet/@author_ref  
      return $tweet/header/date
    }
  </result>' PASSING xml_content RETURNING CONTENT)FROM tweet_CLOB;
\end{minted}

\subsubsection{Listez les utilisateurs qui ont publié un tweet qui a été retwitté au moins deux fois.}
\begin{minted}[linenos, breaklines]{sql}
  SELECT XMLQUERY('for $auteur in /tweeter/users/user
    for $tweet in /tweeter/tweets/tweet
    where $auteur/@id = $tweet/@author_ref and count($tweet/header/retweets/retweet) > 1
    return $auteur' PASSING xml_content RETURNING CONTENT)FROM tweet_CLOB;
\end{minted}

\subsubsection{Pour chaque tweet, listez son contenu et la date de ses deux premières réponses. Rajoutez un element vide \textless nonRetwitted/\textgreater s’il n’a pas été retwitté.}
\begin{minted}[linenos, breaklines]{sql}
  SELECT XMLQUERY('for $tweet in /tweeter/tweets/tweet
  return
  <tweet>
    <contenu>
      { $tweet/body/text}
    </contenu>
    <premieres-reponses>
      { /tweeter/tweets/tweet[@id=$tweet/header/answers/answer[1]/@ref]/header/date }
      { /tweeter/tweets/tweet[@id=$tweet/header/answers/answer[2]/@ref]/header/date }
    </premieres-reponses>
    {
      if (count($tweet/header/retweets/retweet) = 0) then
	<nonRetwitted />
      else()
    }
  </tweet>' PASSING xml_content RETURNING CONTENT)FROM tweet_CLOB;
\end{minted}

\subsection{Binary}
\subsubsection{Créez une liste de paires tweet-auteur, avec chaque paire contenue dans un element result.}
\begin{minted}[linenos, breaklines]{sql}
  SELECT XMLQUERY('for $auteur in /tweeter/users/user 
  for  $tweet in /tweeter/tweets/tweet
  where $auteur/@id = $tweet/@author_ref  
  return
  <result>
  { $tweet }
  { $auteur }
  </result>' PASSING xml_content RETURNING CONTENT)FROM tweet_BINARY;
\end{minted}

\subsubsection{Listez les utilisateurs de la plateforme en ordre alphabétique.}
\begin{minted}[linenos, breaklines]{sql}
  SELECT XMLQUERY('for $user in tweeter/users/user
  order by upper-case($user/user_name/text()) ascending
  return  $user' PASSING xml_content RETURNING CONTENT)FROM tweet_BINARY;
\end{minted}


\subsubsection{Pour chaque utilisateur, listez le nom de l’utilisateur et la date de tous ses tweets, le tout regroupé dansun élément result.}
\begin{minted}[linenos, breaklines]{sql}
  SELECT XMLQUERY('for $auteur in /tweeter/users/user
  return
  <result>
    <nom>
      { $auteur/user_name/text()}
    </nom> 
    {
      for $tweet in /tweeter/tweets/tweet
      where $auteur/@id = $tweet/@author_ref  
      return $tweet/header/date
    }
  </result>' PASSING xml_content RETURNING CONTENT)FROM tweet_BINARY;
\end{minted}

\subsubsection{Listez les utilisateurs qui ont publié un tweet qui a été retwitté au moins deux fois.}
\begin{minted}[linenos, breaklines]{sql}
  SELECT XMLQUERY('for $auteur in /tweeter/users/user
    for $tweet in /tweeter/tweets/tweet
    where $auteur/@id = $tweet/@author_ref and count($tweet/header/retweets/retweet) > 1
    return $auteur' PASSING xml_content RETURNING CONTENT)FROM tweet_BINARY;
\end{minted}

\subsubsection{Pour chaque tweet, listez son contenu et la date de ses deux premières réponses. Rajoutez un element vide \textless nonRetwitted/\textgreater s’il n’a pas été retwitté.}
\begin{minted}[linenos, breaklines]{sql}
  SELECT XMLQUERY('for $tweet in /tweeter/tweets/tweet
  return
  <tweet>
    <contenu>
      { $tweet/body/text}
    </contenu>
    <premieres-reponses>
      { /tweeter/tweets/tweet[@id=$tweet/header/answers/answer[1]/@ref]/header/date }
      { /tweeter/tweets/tweet[@id=$tweet/header/answers/answer[2]/@ref]/header/date }
    </premieres-reponses>
    {
      if (count($tweet/header/retweets/retweet) = 0) then
	<nonRetwitted />
      else()
    }
  </tweet>' PASSING xml_content RETURNING CONTENT)FROM tweet_BINARY;
\end{minted}
