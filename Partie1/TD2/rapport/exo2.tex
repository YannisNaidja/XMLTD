\chapter{XPath : Recettes}
Exprimer en XPath les interrogations suivantes.
Voir document XML en annexe.

\section{Le nom complet de toutes les recettes}
\begin{minted}[linenos, breaklines]{xquery}
  /recettes/recette/@nom
\end{minted}

\section{Les ingrédients de la recette dont le nom court est "Chiffonnade" ;}
\begin{minted}[linenos, breaklines]{xquery}
  /recettes/recette[@nomCourt = "Chiffonnade"]/materiel/ingredient/text()
\end{minted}

\section{Le nom complet des recettes utilisant du “persil” ;}
\begin{minted}[linenos, breaklines]{xquery}
/recettes/recette/materiel/ingredient[contains(text(), "persil")]/parent::materiel/parent::recette/@nom
\end{minted}

\section{(Sans utiliser l’axechild) Le nom complet des recettes utilisant du “persil}
\begin{minted}[linenos, breaklines]{xquery}
/descendant::ingredient[contains(text(), "persil")]/ancestor::recette/@nom
\end{minted}

\section{Le nom complet des recettes ayant plus de deux ingrédients, et contenant des oeuf}
\begin{minted}[linenos, breaklines]{xquery}
/recettes/recette[count(materiel/ingredient) > 2 and materiel/ingredient[contains(text(), "oeuf")]]/@nom
\end{minted}

\section{(Sans utiliser la fonctioncount()) Le nom complet des recettes ayant plus de deux ingrédients, et conte-nant l’ingrédient “huile}
\begin{minted}[linenos, breaklines]{xquery}
/recettes/recette/materiel/ingredient[last() > 2 and contains(text(), "huile")]/parent::materiel/parent::recette/@nom
\end{minted}

\section{La dernière recette du document}
\begin{minted}[linenos, breaklines]{xquery}
/recettes/recette[last()]/@nom
\end{minted}
