\chapter{XQuery : Tweets}
Donner les requêtes XQuery correspondants aux expressions suivantes et évaluer ces expressions dans votredocument XML contenant des Tweets.

\section{Créez une liste de paires tweet-auteur, avec chaque paire contenue dans un element result.}
\begin{minted}[linenos, breaklines]{xquery}
<results> {
  for $auteur in /tweeter/users/user 
  for  $tweet in /tweeter/tweets/tweet
  where $auteur/@id = $tweet/@author_ref  
  return
  <result>
  { $tweet }
  { $auteur }
  </result>  
}
</results>
\end{minted}


\section{Pour chaque utilisateur, listez le nom de l’utilisateur et la date de tous ses tweets, le tout regroupé dansun élément result.}
\begin{minted}[linenos, breaklines]{xquery}
<results> {
  for $auteur in /tweeter/users/user
  return
  <result>
    <nom>
      { $auteur/user_name/text()}
    </nom> 
    {
      for $tweet in /tweeter/tweets/tweet
      where $auteur/@id = $tweet/@author_ref  
      return $tweet/header/date
    }
  </result>
}
</results>
\end{minted}

\section{Listez les utilisateurs qui ont publié un tweet qui a été retwitté au moins deux fois.}
\begin{minted}[linenos, breaklines]{xquery}
<results> {
  for $auteur in /tweeter/users/user
    for $tweet in /tweeter/tweets/tweet
    where $auteur/@id = $tweet/@author_ref and count($tweet/header/retweets/retweet) > 1
    return $auteur
 }
</results>
\end{minted}

\section{Pour chaque tweet, listez son contenu et la date de ses deux premières réponses. Rajoutez un element vide \textless nonRetwitted/\textgreater s’il n’a pas été retwitté.}
\begin{minted}[linenos, breaklines]{xquery}
<results> {
  for $tweet in /tweeter/tweets/tweet
  return
  <tweet>
    <contenu>
      { $tweet/body/text}
    </contenu>
    <premieres-reponses>
      { /tweeter/tweets/tweet[@id=$tweet/header/answers/answer[1]/@ref]/header/date }
      { /tweeter/tweets/tweet[@id=$tweet/header/answers/answer[2]/@ref]/header/date }
    </premieres-reponses>
    {
      if (count($tweet/header/retweets/retweet) = 0) then
	<nonRetwitted />
      else()
    }
  </tweet>  
}
</results>
\end{minted}

\section{Listez les utilisateurs de la plateforme en ordre alphabétique.}
\begin{minted}[linenos, breaklines]{xquery}
<results> {
  for $user in tweeter/users/user
  order by upper-case($user/user_name/text()) ascending
  return  $user
}
</results>
\end{minted}

\section{Listez les tweets contenants l’hashtag “\#I<3XML”.}
\begin{minted}[linenos, breaklines]{xquery}
<results> {
for $tweet in /tweeter/tweets/tweet
where $tweet/body/hashtag[contains(., "#I&lt;3XML")]
return $tweet
}
</results>
\end{minted}

\section{Trouvez le tweet le plus ancien ainsi que le plus recent.}
\begin{minted}[linenos, breaklines]{xquery}
<results> {
  let $minDate := min(/tweeter/tweets/tweet/header/date)
  let $maxDate := max(/tweeter/tweets/tweet/header/date)
  for $tweet in /tweeter/tweets/tweet
  where $tweet/header/date = $minDate or $tweet/header/date = $maxDate
    return $tweet
}
</results>
\end{minted}

\section{Pour chaque tweet ayant des hashtags, retournez le tweet avec la liste de ses hashtag.}
\begin{minted}[linenos, breaklines]{xquery}
<results> {
  for $tweet in /tweeter/tweets/tweet
  where $tweet/body/hashtag
  return
    <result>
      {$tweet}
      <hashtags>
	{$tweet/body/hashtag}
      </hashtags>
    </result>
}
</results>
\end{minted}

\section{Pour chaque tweet ayant des références utilisateur, retournez le tweet avec la liste des références utilisateur.}
\begin{minted}[linenos, breaklines]{xquery}
<results> {
  for $tweet in /tweeter/tweets/tweet
  where $tweet/body/user_ref
  return
    <result>
      {$tweet}
      <user_references>
	{$tweet/body/user_ref}
      </user_references>
    </result>
}
</results>
\end{minted}

\section{Declarez la fonction local:aReponduAuTweet, qui, étant donné un tweet, retourne tous les utilisateurs qui ont répondu au Tweet.}
\begin{minted}[linenos, breaklines]{xquery}
declare function local:aReponduAuTweet($tweet)
{
  for $user in $tweet/parent::tweets/parent::tweeter/child::users/child::user
    where $user/@id = $tweet/parent::tweets/child::tweet[@id = $tweet/header/answers/answer/@ref]/@author_ref
    return $user
};
\end{minted}
