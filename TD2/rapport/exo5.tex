\chapter{Propriétés des requettes XPath}
\section{Reformuler les requêtes suivantes en utilisant exclusivement les axes child, descendant, descendant-or-self, following et following-sibling}
\subsection{Requete 1}
\begin{minted}[linenos, breaklines]{xquery}
//b[parent::a]
\end{minted}

devient

\begin{minted}[linenos, breaklines]{xquery}
//a/b
\end{minted}

\subsection{Requete 2}
\begin{minted}[linenos, breaklines]{xquery}
//a/preceding-sibling::c
\end{minted}

devient

\begin{minted}[linenos, breaklines]{xquery}
//c[following-sibling::a]
\end{minted}

\subsection{Requete 3}
\begin{minted}[linenos, breaklines]{xquery}
//c[preceding::d]
\end{minted}

devient

\begin{minted}[linenos, breaklines]{xquery}
//d/following::c
\end{minted}

\subsection{Requete 4}
\begin{minted}[linenos, breaklines]{xquery}
//b/a/preceding-sibling::c/preceding::d
\end{minted}

devient

\begin{minted}[linenos, breaklines]{xquery}
//d[following::c[following-sibling::a[child::b]]]
\end{minted}

\subsection{Requete 5}
\begin{minted}[linenos, breaklines]{xquery}
/a/b/..//*/./../preceding::d
\end{minted}

devient

\begin{minted}[linenos, breaklines]{xquery}
//d[following::*[child::*]]
\end{minted}

\subsection{Requete 6}
\begin{minted}[linenos, breaklines]{xquery}
//a/ancestor::b/parent::c/child::d/parent::e
\end{minted}

La requette est syntaxiquement correcte mais ne retourne de résultat pour aucun document XML. Voir explication dans la solution au troisième exercice.

\section{Reformuler les requêtes //a/following::b et //a/preceding::b en utilisant les axes descendant-or-self, ancestor, following-sibling et preceding-sibling.}
\subsection{Requete 1}
\begin{minted}[linenos, breaklines]{xquery}
//a/following::b
\end{minted}

devient

\begin{minted}[linenos, breaklines]{xquery}
//a/ancestor-or-self::node()/following-sibling::node()/descendant-or-self::b
\end{minted}

\subsection{Requete 2}
\begin{minted}[linenos, breaklines]{xquery}
//a/preceding::b
\end{minted}

devient

\begin{minted}[linenos, breaklines]{xquery}
//a/ancestor-or-self::node()/preceding-sibling::node()/descendant-or-self::b
\end{minted}


\section{Pour chaque requête définie aux points 1 et 2, proposer un document XML pour lequel la réponse à la requête n’est pas vide, sinon expliquer pourquoi un tel document n’existe pas.}
\subsection{Requete 1 - exercice 1}
\begin{minted}[linenos, breaklines]{xml}
<r>
  <a>
    <b>
    </b>
  </a>
</r>
\end{minted}

\subsection{Requete 2 - exercice 1}
\begin{minted}[linenos, breaklines]{xml}
<r>
  <c>
  </c>
  <a>
  </a>
</r>
\end{minted}

\subsection{Requete 3 - exercice 1}
\begin{minted}[linenos, breaklines]{xml}
<r>
  <a>
    <d>
    </d>
  </a>
  <c>
  </c>
</r>
\end{minted}

\subsection{Requete 4 - exercice 1}
\begin{minted}[linenos, breaklines]{xml}
  <r>
  <a>
    <d>
    </d>
  </a>
  <c>
  </c>
  <a>
    <b>
    </b>
  </a>
</r>
\end{minted}

\subsection{Requete 5 - exercice 1}
\begin{minted}[linenos, breaklines]{xml}
  <r>
  <d>
  </d>
  <a>
    <b>
    </b>
  </a>
</r>
\end{minted}

\subsection{Requete 6 - exercice 1}
Il n'y a aucun document XML correspondant car la fin de la requête est :
\begin{minted}[linenos, breaklines]{xquery}
c/child::d/parent::e
\end{minted}
Or un noeud de type c ne peut pas avoir d'enfant de type d qui ait un parent e car un noeud n'a qu'un seul parent.

\subsection{Requete 1 - exercice 2}
\begin{minted}[linenos, breaklines]{xml}
<r>
  <s>
    <a></a>
    <b>b1</b>
  </s>
  <b>b2</b>
</r>
\end{minted}

\subsection{Requete 2 - exercice 2}
\begin{minted}[linenos, breaklines]{xml}
<r>
  <b>b2</b>
  <s>
    <b>b1</b>
    <a></a>
  </s>
</r>
\end{minted}

\section{Donner un document XML pour lequel la requête //r[a[1] = a[2]] n’est pas vide sans que les éléments comparés soient strictement identiques.}
\begin{minted}[linenos, breaklines]{xml}
<r>
  <a>
    abcdef
  </a>
  <a>
    <b>
      abc
    </b>
    <b>
      def
    </b>
  </a>
</r>
\end{minted}
Les deux noeuds de type a sont égaux en XPath car ils partagent la sous-chaine abc, pourtant il n'y a pas isomorphisme entre les deux sous-arbres et les noeuds textes ne sont pas identiques.

\section{Est il vrai que, dans le cadre du langage XPath, si X = Y et Y = Z alors X = Z?}
Non, en XPath, l'égalité n'est pas transitive.

Voici un contre exemple :
\begin{minted}[linenos, breaklines]{xquery}
<root>
  <x>
    abc
  </x>
  <y>
    <y1>
      abc
    </y1>
    <y2>
      def
    </y2>
  </y>
  <z>
    def
  </z>
</root>
\end{minted}

Dans ce document XML, les égalités x = y et y = z sont vraie en XPath mais l'égalité x = z est fausse
