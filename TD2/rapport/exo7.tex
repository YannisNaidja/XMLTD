\chapter{XQuery : Trains}
Donner des expressions XQuery pour les requêtes suivantes.

Voir document XML en annexe.

\section{Le numéro des trains possédant une voiture-bar.}
\begin{minted}[linenos, breaklines]{xquery}
  <results> {
    for $train in /gare/train
    where $train/voiture/bar
    return
    <numero>
    {$train/@numero}
    </numero>
  }
  </results>
\end{minted}

\section{Le nom des usages ayant au moins une réservation}.
\begin{minted}[linenos, breaklines]{xquery}
  <results> {
    let $resa :=  /gare/train/voiture/resa
    for $user in /gare/usager
    where $user/@id = $resa/@id
    return
    <usager>
    { string($user/@nom) }
    </usager>
  }
  </results>
\end{minted}

\section{La reservation avec le plus grand identifiant (dans l’ordre lexicographique).}
\begin{minted}[linenos, breaklines]{xquery}
  <results> {
    (for $resa in /gare/train/voiture/resa
    order by $resa/@numero descending
    return $resa)[1]
  }
  </results>
\end{minted}

\section{Le numéro des trains dont au moins 2 places sont réservées.}
\begin{minted}[linenos, breaklines]{xquery}
<results> {
  for $voiture in /gare/train/voiture
  where (count($voiture/resa) >= 2)
  return
    <train>
    { string($voiture/parent::train/@numero) }
    </train>
}
</results>
\end{minted}

\section{Le nom des personnes ayant réservé exactement deux fois.}
\begin{minted}[linenos, breaklines]{xquery}
<results> {
  for $user in /gare/usager
  let $count := count(/gare/train/voiture/resa[@id = $user/@id]) return
    if ($count = 2) then
    <usager>
      { string($user/@nom) }
    </usager>
    else()
}
</results>
\end{minted}

\section{Les usagers n’ayant effectué aucune réservation.}
\begin{minted}[linenos, breaklines]{xquery}
  <results> {
    let $resa :=  /gare/train/voiture/resa
    for $user in /gare/usager
    where not($user/@id = $resa/@id)
    return $user
  }
  </results>
\end{minted}
