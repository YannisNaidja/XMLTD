\chapter{Stockage schema-aware : Verical-Edge vs Monet}
\section{À partir de la DTD pour les bâtiments présentée dans l’énoncé du TD1, définir un schéma de stockagerelationnel suivant la méthode présentée en cours.}
\subsection{DTD batiment}
\begin{minted}[linenos, breaklines]{dtd}
<!DOCTYPE batiment [ 
<!ELEMENT batiment (etage)+ >
<!ELEMENT etage (description,(bureau+|salle+)) >
<!ELEMENT description (#PCDATA) >
<!ELEMENT bureau (code, personne*) >
<!ELEMENT code (#PCDATA) >
<!ELEMENT personne (#PCDATA) >
<!ELEMENT salle (nombrePlaces) >
<!ELEMENT nombrePlaces (#PCDATA) >]>
\end{minted}

\subsection{Suppression des symboles +}
\begin{minted}[linenos, breaklines]{dtd}
<!DOCTYPE batiment [ 
<!ELEMENT batiment (etage*,etage) >
<!ELEMENT etage (description,((bureau*,bureau)|(salle*,salle)) >
<!ELEMENT description (#PCDATA) >
<!ELEMENT bureau (code, personne*) >
<!ELEMENT code (#PCDATA) >
<!ELEMENT personne (#PCDATA) >
<!ELEMENT salle (nombrePlaces) >
<!ELEMENT nombrePlaces (#PCDATA) >]>
\end{minted}

\subsection{Suppression de l'ordre et des correlations}
\begin{minted}[linenos, breaklines]{dtd}
<!DOCTYPE batiment [ 
<!ELEMENT batiment (etage* | etage) >
<!ELEMENT etage (description | bureau* | bureau | salle* | salle)) >
<!ELEMENT description (#PCDATA) >
<!ELEMENT bureau (code | personne*) >
<!ELEMENT code (#PCDATA) >
<!ELEMENT personne (#PCDATA) >
<!ELEMENT salle (nombrePlaces) >
<!ELEMENT nombrePlaces (#PCDATA) >]>
\end{minted}

\subsection{Simplifications}
r|r* est équivalent à r*
\begin{minted}[linenos, breaklines]{dtd}
<!DOCTYPE batiment [ 
<!ELEMENT batiment (etage*) >
<!ELEMENT etage (description | bureau* | salle*) >
<!ELEMENT description (#PCDATA) >
<!ELEMENT bureau (code | personne*) >
<!ELEMENT code (#PCDATA) >
<!ELEMENT personne (#PCDATA) >
<!ELEMENT salle (nombrePlaces) >
<!ELEMENT nombrePlaces (#PCDATA) >]>
\end{minted}

\subsection{Représentation sous forme de graphe}
Voir graphe en annexe

\subsection{Relations}
\begin{itemize}
\item batiment(\underline{batimentID: integer}, flagRoot: integer)
\item etage(\underline{etageID: integer}, batimentID: integer, description: string)
\item bureau(\underline{bureauID: integer}, etageID: integer, code: string)
\item personne(\underline{personneID: integer}, bureauID: integer)
\item salle(\underline{salleID: integer}, etageID: integer, nombreDePlace: integer)
\end{itemize}
